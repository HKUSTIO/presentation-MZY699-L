\documentclass[aspectratio=169]{beamer}  % 16:9 aspect ratio

% Use a clean theme as base
\usetheme{default}
\usecolortheme{default}

% Custom colors from HKUST logo
\definecolor{hkustblue}{RGB}{0, 51, 119}    % Navy blue from logo
\definecolor{hkustgold}{RGB}{180, 141, 61}  % Golden brown from logo
\definecolor{lightgray}{RGB}{236, 240, 241}

% Customize the appearance
\setbeamercolor{structure}{fg=hkustblue}
\setbeamercolor{background canvas}{bg=white}
\setbeamercolor{normal text}{fg=hkustblue}
\setbeamercolor{frametitle}{fg=hkustblue,bg=white}
\setbeamercolor{itemize item}{fg=hkustgold}
\setbeamercolor{itemize subitem}{fg=hkustgold}
\setbeamercolor{block title}{fg=white,bg=hkustblue}
\setbeamercolor{block body}{fg=hkustblue,bg=lightgray}
\setbeamercolor{title}{fg=hkustblue}
\setbeamercolor{subtitle}{fg=hkustgold}

% Remove navigation symbols
\setbeamertemplate{navigation symbols}{}

% Customize frame title
\setbeamertemplate{frametitle}{
    \vspace*{0.5cm}
    \insertframetitle
    \vspace*{0.2cm}
    \begin{beamercolorbox}[wd=\paperwidth,ht=0.2pt]{structure}
    \end{beamercolorbox}
}

% Customize itemize bullets
\setbeamertemplate{itemize item}{\small\raise0.5pt\hbox{\textbullet}}
\setbeamertemplate{itemize subitem}{\tiny\raise1.5pt\hbox{\textbullet}}

% Packages
\usepackage{graphicx}
\usepackage{amsmath}
\usepackage{hyperref}

% Title page information
\title{Foundation of demand estimation}
\subtitle{Theoretical and Empirical Study}
\author{Ma Zhiyao}
\institute{Hong Kong University of Science and Technology}
\date{\today}

\begin{document}

% Title page
\begin{frame}
    \titlepage
\end{frame}

% Table of contents
\begin{frame}{Outline}
    \tableofcontents
\end{frame}

% Section 1
\section{Introduction}
\begin{frame}{1.Introduction: Why estimate demand?}
    \begin{itemize}
        \item Measuring Market Power

        
        Accurate demand elasticities are essential for computing markups and identifying firms’ pricing power.
        \item Estimating Marginal Costs and Markups


        Structural demand estimates are required to back out marginal costs using firms' first-order conditions in equilibrium models.
        \item Simulating Counterfactual Policies


        Demand models enable simulations of mergers, taxes, price regulations, and other policy interventions on market outcomes and consumer welfare.
    \end{itemize}
\end{frame}

\begin{frame}{1.Introduction: Challenges}
    \begin{itemize}
        \item The endogeneity of price
            \begin{align}
            Q &= D(X, P, U) \tag{1.1} \\
            P &= C(W, Q, V). \tag{1.2}
                \end{align}
        \item Demand for any one good generally depends on more than one latent demand shocks
    \end{itemize}
\end{frame}

\begin{frame}{1.Introduction: Challenges}
    \begin{itemize}
        \item LATE(Local average treatment effect)
            \begin{align}
            Q = S(W, P, V), 
\frac{\partial P^*(X, W, U, V, \tau)}{\partial \tau} =
\frac{ \frac{\partial S(W, P, V)}{\partial P} }
{ \frac{\partial S(W, P, V)}{\partial P} + \left| \frac{\partial D(X, P, U)}{\partial P} \right| }.
\tag{1.3}
    \end{align}
        \item Fixed effect and control function
    \end{itemize}
\end{frame}

\begin{frame}{1.Introduction: Solution}
    \begin{itemize}
        \item Use structural instrumental variables (e.g., BLP instruments, cost shifters) and moment conditions like
        \begin{equation}
\mathbb{E}[\xi_{jt} \cdot Z_{jt}] = 0 \tag{1.4}
\end{equation}
        \item Estimate a system-wide discrete choice model (e.g., multinomial logit, random coefficients logit) with full substitution structure
    \end{itemize}
\end{frame}

% Section 2
\section{Models}
\begin{frame}{2.Models: RUM}
    \begin{block}{Random utility model}
        \begin{equation*}
    u_{ij} = V_{ij} + \varepsilon_{ij}
\end{equation*}
    \end{block}
    
    \begin{itemize}
        \item 
        \begin{align}
q_{ij} &= \mathbf{1} \left\{ u_{ij} \geq u_{ik} \ \forall k \in \{0, 1, \ldots, J_i\} \right\}\tag{2.1}. \\[1em]
s_{ij} &= \mathbb{E} \left[ q_{ij} \mid J_i, \chi_i \right]\tag{2.2} \\ 
&= \int_{A_{ij}} dF_u \left( u_{i0}, u_{i1}, \ldots, u_{iJ_i} \mid J_i, \chi_i \right)\tag{2.2}. \\[1em]
A_{ij} &= \left\{ \left( u_{i0}, u_{i1}, \ldots, u_{iJ_i} \right) \in \mathbb{R}^{J_i + 1} : u_{ij} \geq u_{ik} \ \forall k \right\}\tag{2.3}.
\end{align}
    \end{itemize}
\end{frame}

\begin{frame}{2.Models: Canonical Model}
    \begin{center}
        \begin{align}
            u_{ijt} &= x_{jt} \beta_{it} - \alpha_{it} p_{jt} + \xi_{jt} + \epsilon_{ijt}, && \text{for } j > 0 \tag{2.4} \\
            u_{i0t} &= \epsilon_{i0t} \tag{2.5}
        \end{align}

        \vspace{1em}

        \begin{equation}
            s_{jt} = \int \frac{e^{x_{jt} \beta_{it} - \alpha_{it} p_{jt} + \xi_{jt}}}{\sum_{k=0}^{J_t} e^{x_{kt} \beta_{it} - \alpha_{it} p_{kt} + \xi_{kt}}} \, dF(\alpha_{it}, \beta_{it}; t)
            \tag{2.6}
        \end{equation}

        \vspace{1em}

        \begin{equation}
            \beta_{it}^{(k)} = \beta_0^{(k)} + \beta_{\nu}^{(k)} \nu_{it}^{(k)} + \sum_{\ell=1}^{L} \beta_{d}^{(\ell, k)} d_{i \ell t}
            \tag{2.7}
        \end{equation}
        \begin{equation}
            \ln(\alpha_{it}) = \alpha_0 + \alpha_y y_{it} + \alpha_{\nu} \nu_{it}^{(0)}\tag{2.8},
        \end{equation}
    \end{center}
\end{frame}

\begin{frame}{2.Models:Canonical Model}
    \begin{itemize}
    \item Random coefficients are introduced in discrete choice models to address the limitations of models with only fixed coefficients.
     \end{itemize}
        \begin{align}
u_{ijt} &= x_{jt} \beta_0 - \alpha p_{jt} + \xi_{jt} + \epsilon_{ijt} \tag{2.9} \\
\delta_{jt} &= x_{jt} \beta_0 - \alpha p_{jt} + \xi_{jt} \tag{2.10} \\
u_{ijt} &= \delta_{jt} + \epsilon_{ijt} \tag{2.11}
\end{align}
\end{frame}

% Section 3
\section{Market level}
\begin{frame}{3.Market level:Setting}
In many applications, the key data are observed at the market level. Typically, one observes:
\begin{itemize}
    \item The number of goods \( J_t \) available to consumers in each market \( t \);
    \item Their prices and other observable characteristics \( p_t, x_t \);
    \item Their observed market shares \( \tilde{s}_{jt} \), typically measured as the total quantity of good \( j \) sold in market \( t \) divided by the number of consumers (e.g., households) in that market;
    \item The distribution of consumer characteristics \( (d_{it}, y_{it}) \) in each market;
    \item Possibly, additional variables \( w_t \) (e.g., cost shifters) that might serve as appropriate instruments.
\end{itemize}
\end{frame}

\begin{frame}{3.Market level: BLP}
    \begin{itemize}
        \item \textbf{Goal:} Estimate demand parameters \( \theta = (\theta_1, \theta_2) \)
        \item \( \theta_1 = (\alpha_0, \beta_0) \): linear parameters
        \item \( \theta_2 = (\beta_d, \beta_\nu) \): nonlinear parameters in random coefficients
        \item \textbf{Key steps:}
        \begin{enumerate}
            \item Fix a trial \( \theta \)
            \item Invert market shares \( \tilde{s}_{jt} \) to recover \( \delta_{jt}(\theta_2) \)
            \item Recover unobserved \( \xi_{jt}(\theta) \)
            \item Evaluate GMM objective
        \end{enumerate}
    \end{itemize}
\end{frame}

\begin{frame}{3.Market level: BLP}
    \textbf{GMM Objective:}
    \begin{equation}
        \min_{\theta} \quad g(\xi(\theta))' \Omega g(\xi(\theta)) \tag{3.1}
    \end{equation}
    \textbf{Moment condition:}
    \begin{equation}
        g(\xi(\theta)) = \frac{1}{N} \sum_{j,t} \xi_{jt}(\theta) z_{jt} \tag{3.2}
    \end{equation}
    \textbf{Constructing \( \xi_{jt}(\theta) \):}
    \begin{equation}
        \xi_{jt}(\theta) = \delta_{jt}(\theta_2) - x_{jt}'\beta + \alpha p_{jt} \tag{3.3}
    \end{equation}
\end{frame}

\begin{frame}{3.Market level: BLP}
    \textbf{Matching predicted to observed shares:}
    \begin{equation}
        \log(\tilde{s}_{jt}) = \log(\sigma_j(\delta_t, x_t, \theta_2, J_t)) \tag{3.4}
    \end{equation}
    \textbf{Predicted share function:}
    \begin{equation}
        \sigma_j(\cdot) = \int \frac{\exp[\delta_{jt}(\theta_2) + x_{jt}'\tilde{\beta}]}{1 + \sum_k \exp[\delta_{kt}(\theta_2) + x_{kt}'\tilde{\beta}]} f_{\tilde{\beta}}(\tilde{\beta}|\theta_2) d\tilde{\beta} \tag{3.5}
    \end{equation}
    \textbf{Solve for \( \delta_{jt} \) such that:}
    \[ \sigma_j(\delta_t, x_t, \theta_2, J_t) = \tilde{s}_{jt} \]
\end{frame}

\begin{frame}{3.Market level: BLP}
\begin{itemize}
    \item \textbf{Cost Shifters}
    \begin{itemize}
        \item \textit{Definition:} Supply-side variables like input costs, taxes, or tariffs.
        \item \textit{Advantage:} Clearly exogenous; commonly used in structural models.
        \item \textit{Limitation:} Limited variation or measurement error possible.
    \end{itemize}
    
    \item \textbf{BLP Instruments}
    \begin{itemize}
        \item \textit{Definition:} Characteristics of competing products (e.g., rival’s features).
        \item \textit{Advantage:} Exploits substitution across products; widely accepted.
        \item \textit{Limitation:} Requires independence from product-specific unobservables ($\xi_{jt}$).
    \end{itemize}
\end{itemize}
\end{frame}

\begin{frame}{3.Market level: BLP}
\begin{itemize}
    \item \textbf{Waldfogel-Fan Instruments}
    \begin{itemize}
        \item \textit{Definition:} Demographics from nearby markets (e.g., average income).
        \item \textit{Advantage:} Capture spatial spillovers; useful in zone pricing contexts.
        \item \textit{Limitation:} Depend on pricing geography and data resolution.
    \end{itemize}

    \item \textbf{Exogenous Market Structure Shifts}
    \begin{itemize}
        \item \textit{Definition:} Firm entry/exit, mergers, or ownership changes.
        \item \textit{Advantage:} Change competitive intensity; often impactful.
        \item \textit{Limitation:} Need exogeneity and observable events.
    \end{itemize}

    \item \textbf{Optimal Instruments}
    \begin{itemize}
        \item \textit{Definition:} Function of observables maximizing moment efficiency.
        \item \textit{Form:} \[ z^*_{jt} = \psi^{-1}_{jt} \mathbb{E} \left[ \frac{\partial \xi_{jt}(\theta)}{\partial \theta} \middle| z_{jt} \right] \]
        \item \textit{Limitation:} Infeasible without full distributional knowledge of $\xi$.
    \end{itemize}
\end{itemize}
\end{frame}

\begin{frame}{3.Motivation for Adding Supply Side}
\begin{itemize}
    \item Demand-only estimation can identify \( \theta_1, \theta_2 \), but precision may be low.
    \item Instrumental variables may be weak or hard to justify.
    \item Supply-side modeling adds additional moment conditions.
    \item Enables interpretation of marginal cost and markup.
    \item Supports policy counterfactuals (mergers, taxes, pricing rules).
\end{itemize}
\end{frame}

\begin{frame}{3.How to Add Supply Side Estimation}
\begin{enumerate}
    \item Estimate demand parameters \( \theta = (\theta_1, \theta_2) \) using observed shares and instruments.
    \item Recover mean utility \( \delta_{jt} \) via share inversion.
    \item Compute implied markups from demand elasticities:
    \[
    mc_{jt} = p_{jt} - \text{markup}_{jt}(\theta)
    \]
    \item Assume a marginal cost function:
    \[
    mc_{jt} = w_{jt}' \gamma + \omega_{jt}
    \]
    \item Construct additional GMM moments:
    \[
    \mathbb{E}[\omega_{jt}(\theta, \gamma) \cdot w_{jt}] = 0
    \]
\end{enumerate}
\end{frame}

\begin{frame}{3.What Does Supply Side Estimation Give Us?}
\begin{itemize}
    \item \textbf{Improved identification of \(\theta_1\)}:
    \begin{itemize}
        \item Exploits cost-side variation.
        \item Less dependent on demand-side instruments.
    \end{itemize}
    \item \textbf{Estimation of cost structure parameters \(\gamma\)}:
    \begin{itemize}
        \item Provides insight into marginal cost behavior.
        \item Useful for merger simulations and policy evaluation.
    \end{itemize}
    \item \textbf{Full structural interpretation}:
    \begin{itemize}
        \item Enables joint analysis of demand and supply.
        \item Supports realistic counterfactuals.
    \end{itemize}
\end{itemize}
\end{frame}

% Section 4
\section{Working paper}
\begin{frame}{4.Working paper}
\centering

\large\textbf{Identification in Differentiated Products Markets Using Market Level Data} \\[1.5em]

\large Steven T. Berry \\
Philip A. Haile \\[1em]

\small
\textbf{Abstract } \\[0.8em]

This paper studies \textit{nonparametric identification} in differentiated products markets using only market-level data. It allows for \textit{random utility models}, endogenous product characteristics, and unobserved heterogeneity on both demand and supply sides. \\[0.6em]

Two identification strategies are proposed: one based on \textit{instrumental variables}, and another using a \textit{change-of-variables approach}. \\[0.6em]

The results formalize \textbf{Bresnahan’s (1982)} test for distinguishing between oligopoly models.

\end{frame}

\begin{frame}{4.Assumption 1a and 1b: Structure of Utility}
\textbf{Assumption 1a:} For all $\tilde{u}t \in \mathcal{U}$, \quad $\tilde{u}{it}(x_{jt}, p_{jt}, \xi_{jt}) = \mu_t(\delta_{jt}, x_{jt}^{(2)}, p_{jt})$ \quad where $\mu_t$ is strictly increasing in $\delta_{jt}$.

\vspace{0.3cm}
\textbf{Assumption 1b:} There exists a monotonic function $\Gamma_t$ such that \
\quad $\Gamma_t(\tilde{u}{it}(x{jt}, p_{jt}, \xi_{jt})) = \mu_t(\delta_{jt}, x_{jt}^{(2)}, \omega_{it}) - p_{jt}$

\vspace{0.3cm}
\textit{Motivation:} These assumptions impose quasilinearity and index restrictions to simplify the indirect utility specification and facilitate identification through monotonic transformations.
\end{frame}

\begin{frame}{4.Assumptions 2 and 3: Market Behavior and Substitution}
\textbf{Assumption 2:} Market shares $s_j$ are strictly positive for all $j$ at any observed ${x_{kt}, p_{kt}, \xi_{kt}}_{k \in \mathcal{J}}$.

\vspace{0.3cm}
\textbf{Assumption 3 (Connected Substitutes):} The directed graph of substitutes $\Sigma(\mathcal{J})$ is strongly connected: for all ${x_{jt}, p_{jt}, \xi_{jt}}_{j \in \mathcal{J}}$, there exists a substitution path between any pair $j, j'$.

\vspace{0.3cm}
\textit{Motivation:} These ensure observability of all products and interconnectedness of substitution patterns, avoiding isolated or unused products.
\end{frame}

\begin{frame}{4.Assumptions 4 and 5: Instrument Validity and Completeness}
\textbf{Assumption 4:} \quad $\mathbb{E}[\xi_{jt} \mid \tilde{z}_t, x_t] = 0$ \quad for all $j$ (instrument exogeneity)

\vspace{0.3cm}
\textbf{Assumption 5:} \quad For any $B(s_t, p_t)$ with finite expectation, \\n$\mathbb{E}[B(s_t, p_t) \mid \tilde{z}_t, x_t] = 0 \Rightarrow B(s_t, p_t) = 0$ almost surely

\vspace{0.3cm}
\textit{Motivation:} These conditions validate IV strategies and ensure the model is rich enough (completeness) to identify structural parameters from observable variation.
\end{frame}

\begin{frame}{4.Proof of Theorem 1 (Identification of \(\xi_{jt}\))}
\textbf{Step-by-step:}
\begin{itemize}
  \item Equation: $x_{jt} + \xi_{jt} = \sigma_j^{-1}(s_t, p_t)$
  \item Take conditional expectation: \
    $\mathbb{E}[\xi_{jt} | \tilde{z}_t, x_t] = \mathbb{E}[\sigma_j^{-1}(s_t, p_t) | \tilde{z}_t, x_t] - x_{jt}$
  \item By Assumption 4: $\mathbb{E}[\xi_{jt} | \tilde{z}_t, x_t] = 0$, so:
    \[ \mathbb{E}[\sigma_j^{-1}(s_t, p_t) | \tilde{z}_t, x_t] - x_{jt} = 0 \]
  \item Suppose an alternative function $\bar{\sigma}_j^{-1}$ satisfies:
    \[ \mathbb{E}[\bar{\sigma}_j^{-1}(s_t, p_t) - x_{jt} | \tilde{z}_t, x_t] = 0 \]
  \item Define $B(s_t, p_t) = \sigma_j^{-1}(s_t, p_t) - \bar{\sigma}_j^{-1}(s_t, p_t)$, so:
    \[ \mathbb{E}[B(s_t, p_t) | \tilde{z}_t, x_t] = 0 \]
  \item By Assumption 5 (completeness): $B = 0$ a.s. \Rightarrow $
    $\bar{\sigma}_j^{-1} = \sigma_j^{-1}$
    
\end{itemize}
\end{frame}

\begin{frame}{4.Assumptions 6--8 (Supply Side Identification)}

\textbf{Assumption 6.} \textit{Support Condition ("Covering Condition")}\\
\smallskip
\quad $\text{supp}\ p_t \mid \left\{x_{jt}, \xi_{jt}\right\}_{j \in \mathcal{J}} \ \supseteq \ \text{supp} \left( \mu_1(\delta_{1t}, \omega_{it}), \ldots, \mu_J(\delta_{Jt}, \omega_{it}) \right)_t \mid \left\{x_{jt}, \xi_{jt} \right\}_{j \in \mathcal{J}}$\\
\smallskip
This condition ensures that observed prices are rich enough to "cover" variation in consumers’ willingness to pay. 

\bigskip
\textbf{Assumption 7.} \textit{Differentiability of Market Shares}\\
\smallskip
\quad $\sigma_j(\delta_t, p_t)$ is continuously differentiable w.r.t. $p_k$ for all $j, k \in \mathcal{J}$.\\
\smallskip
This is needed to apply the first-order conditions and identify supply-side behavior.

\bigskip
\textbf{Assumption 8.} \textit{Structure of Marginal Costs and Residual Revenue Function}\\
\begin{itemize}
    \item[(i)] $mc_j(q_{jt}, \zeta_{jt})$ is strictly increasing in $\zeta_{jt}$
    \item[(ii)] $u_j(\delta_{jt}, p_{jt}, \omega_{it})$ is strictly decreasing in $p_{jt}$
    \item[(iii)] There exists a function $\psi_j$ such that:
    \[
    mc_j(M_t s_{jt}, \zeta_{jt}) = \psi_j(s_t, M_t, D_t(s_t, p_t), p_t)
    \]
\end{itemize}
\end{frame}

\begin{frame}{4.Theorems}

\textbf{Theorem 2.} \textit{Under Assumptions 1b and 2--6, the joint distribution of}
\[ (v_{i1t}, \ldots, v_{iJt}) \text{ conditional on any } \{x_{kt}, p_{kt}, \xi_{kt}\}_{k \in \mathcal{J}} \in \chi^{\mathcal{J}} \text{ is identified.} \]

\vspace{0.5cm}

\textbf{Theorem 3.} \textit{Suppose that Assumptions 1a, 2--5, 7 and 8 hold. Then for all $j$}
\begin{itemize}
    \item[(i)] $\eta_{jt}$ is identified for all $t$, and
    \item[(ii)] if $\psi_j$ is known, the function $mc_j(q_{jt}, \zeta_{jt})$ is identified on the support of $(q_{jt}, \zeta_{jt})$.
\end{itemize}
\end{frame}

\begin{frame}{4.Assumptions 9--13 for Identification}
  \textbf{Assumption 9.} There is a unique vector of equilibrium prices associated with any $(\delta, \zeta)$. \\
  \vspace{0.2cm}
  \textbf{Assumption 10.} The random variables $(\xi_1, \ldots, \xi_J, \eta_1, \ldots, \eta_J)$ have a positive joint density $f_{\xi,\eta}$ on $\mathbb{R}^{2J}$. \\
  \vspace{0.2cm}
  \textbf{Assumption 11.} The vector function $\left(\sigma_1^{-1}, \ldots, \sigma_J^{-1}, \pi_1^{-1}, \ldots, \pi_J^{-1} \right)'$ has continuous partial derivatives and nonzero Jacobian determinant. \\
  \vspace{0.2cm}
  \textbf{Assumption 12.} $(x_t, z_t) \perp\!\!\!\perp (\xi_t, \eta_t)$. \\
  \vspace{0.2cm}
  \textbf{Assumption 13.} $\operatorname{supp}(x_t, z_t) = \mathbb{R}^{2J}$.
\end{frame}

\begin{frame}{4.Theorem 4 and 5: Identification in Differentiated Product Markets}

\textbf{Theorem 4.} \textit{Suppose Assumptions 1a, 2, 3, and 8--13 hold. Then for all $j$
\begin{itemize}
    \item[(i)] $\xi_{jt}$ is identified for all $t$, and
    \item[(ii)] the function $s_j\left(\{x_{kt}, p_{kt}, \xi_{kt}\}_{k \in \mathcal{J}}\right)$ is identified on $\mathcal{X}^{\mathcal{J}}$.
\end{itemize}}

\vspace{1em}

\textbf{Theorem 5.} \textit{Suppose Assumptions 1b, 2, 3, 6, and 8--13 hold. Then the joint distribution of $\left(v_{i1t}, \ldots, v_{iJt}\right)$ conditional on any $\{x_{kt}, p_{kt}, \xi_{kt}\}_{k \in \mathcal{J}} \in \mathcal{X}^{\mathcal{J}}$ is identified.}

\end{frame}

\begin{frame}{4.Discriminating Between Oligopoly Models}
\textbf{Motivation:} Can different models of firm conduct (e.g., marginal cost pricing vs. monopolistic pricing) be distinguished empirically?

\textbf{Main Idea:} Use variations in the residual marginal revenue function $\psi_j$ to test supply models.

\textbf{Key Setup:}
\begin{itemize}
  \item Under Assumption 8, marginal cost satisfies:
  \begin{equation*}
    mc_j(q_{jt}, \zeta_{jt}) = \psi_j(s_t, D_t(s_t, p_t), p_t)
  \end{equation*}
  \item Let $\psi_j^0$ and $\psi_j^1$ denote marginal revenue functions under the null and alternative hypotheses, respectively.
\end{itemize}

\textbf{Identification via Rotations:}
\begin{itemize}
  \item Suppose market conditions change such that $\psi_j$ rotates (e.g., due to instrument variation).
  \item If null hypothesis is true: implied marginal cost $mc_j^0$ should remain invariant.
  \item If under rotation, marginal cost is instead $mc_j^1 \neq mc_j^0$: null is falsified.
\end{itemize}
\end{frame}

\begin{frame}{4.Discriminating Between Oligopoly Models (Visual)}
\textbf{Illustration of Model Testing:}

\begin{center}
    \includegraphics[width=0.42\textwidth]{figure/figure2.png}
    \hfill
    \includegraphics[width=0.42\textwidth]{figure/figure3.png}

    \vspace{0.5em}
    {\small Figure 2: Mapping of $E_t$ to marginal cost; Figure 3: Rotation rejects false null.}
\end{center}
\end{frame}



\end{document}
